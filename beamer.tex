%!TEX program = lualatex
\documentclass[10pt,mathserif]{beamer}

\usepackage{luatexja}
\usepackage{luatexja-fontspec}
\setmainjfont[
    RawFeature={instance=Regular},
    BoldFont=Noto Sans CJK SC,
    BoldFeatures={RawFeature={instance=Bold}},
	Renderer=Harfbuzz,
]{Noto Sans CJK SC}

\setmonofont[
    RawFeature={instance=Regular},
    BoldFont=Noto Sans Mono,
    BoldFeatures={RawFeature={instance=Bold}},
    ItalicFont=Noto Sans Mono,
    ItalicFeatures={RawFeature={instance=Italic}},
	Renderer=Harfbuzz,
]{Noto Sans Mono}

\newfontfamily\light{Noto Sans Mono}[
	RawFeature={instance=ExtraLight},
	Renderer=Harfbuzz,
]

\definecolor{xdublue}{RGB}{0,65,130}

\usepackage{listings}
\lstset{
basicstyle=\small\ttfamily,
keywordstyle=\color{xdublue},
numbers=left,
numberstyle=\tiny\light,
commentstyle=\light,
frame=leftline,
tabsize=4
}

\lstdefinestyle{term}
{basicstyle=\ttfamily, numbers=none, frame=single, breaklines=true,
moredelim={[is][keywordstyle]{@@}{@@}}}

\newcommand{\lstcode}[1] { \lstinputlisting[language=C++]{code/#1} }
\newcommand{\lstterm}[1] { \lstinputlisting[style=term]{code/#1} }

\usepackage{ulem}

\usetheme[xdblue]{XDUstyle}

\title{C++ Subset for Comptitive Programming}
\institute{西安电子科技大学程序设计竞赛实训基地}
\author{席若尧}
\date{2022 年 5 月 5 日}
	
\begin{document}%
{\xdbg \frame[plain,noframenumbering]{\titlepage}}

\begin{frame}{内容}
	\tableofcontents[hideallsubsections]
\end{frame}

\section{为什么用 C++}
\sectionpage

\begin{frame}{为什么不用 C}
	\begin{itemize}
		\item Q: 我只会 C,能打 ACM 吗?
			\pause
		\item 大佬: 能,用 C 连 Linux 内核都能写出来,还写不出几个算法题?
			\pause
		\item 我: 但是你能手写红黑树的 5 种插入和 7 种删除吗……
			Linux 内核的红黑树代码 334 行。
	\end{itemize}
\end{frame}

\begin{frame}{为什么不用 Java}
	\begin{itemize}
		\item Q: 人家大公司都用 Java,我用这个东西打比赛也没问题罢?
			\pause
		\item 大佬: 没问题,我只会 Java 都打到 CF LGM 了。
			\pause
		\item 但是 Java 这个东西在用得好的人和一般人手里效率差距很大,
			比如要用 Java 打比赛首先需要写个高效率 I/O 库,如果你直接用
			Java 自带的 I/O 连 A+B 都能超时。
		\item 随着 JVM 的进步,目前多数评测环境已经不再为 Java 单独开时限。
			\pause
		\item Kotlin 基本就是 Java 套壳,同理。
	\end{itemize}
\end{frame}

\begin{frame}{为什么不用 Python}
	\begin{itemize}
		\item Q: 学术界 Python 都能算神经网络,
			我用这个东西打比赛也没问题罢?
		\item 人学术界是用 Python 封装 C/C++ 写的运算库,
			比赛时候没有运算库给你用,而且目前评测系统不支持你自己写运算库。
		\item 只用纯 Python 的话就算你再强,写大多数题目都会超时。
			当然有的特殊题目 (高精度,简单的字符串处理等) 是可以用的……
	\end{itemize}
\end{frame}

\begin{frame}{为什么不用 XXX}
	\begin{itemize}
		\item Pascal, Go, Rust, OCamal (@zzs), MATLAB, Brainf**k, ……
		\item 因为 (正式) 比赛用不了
		\item 总而言之,是因为 C++ 的语言特性和标准库提供的功能比较适合
			这个比赛
	\end{itemize}
\end{frame}

\section{一些 C 语言老师应该讲但是没讲/讲错的东西}
\sectionpage

\begin{frame}{误区}
	\begin{itemize}
		\item Q: C 是 C++ 的子集?
			\pause
		\item A: 并不是
			\begin{itemize}
				\item C++ 具有更严格的类型检查要求
				\item \lstinline{inline} 等一些东西虽然看起来是一样的,
					但是语义上有微妙的差别
			\end{itemize}
	\end{itemize}
\end{frame}

\begin{frame}{误区}
	\begin{itemize}
		\item 如果你不确定你的程序会如何表现,就跑一下试试
			\pause
		\item 这相当于说:如果你不确定一个化学反应是什么样的,
			就直接在试管里试一试
			\pause
		\item 使用未初始化值、数组越界、带符号整数溢出等都具有不确定的行为,
			这些在不同的机器上,使用不同的编译器或编译选项时,
			甚至重复运行时都可能产生出乎意料的不同结果
	\end{itemize}
\end{frame}

\section{竞赛中可能用到的 C++ 特性}

\begin{frame}{一个 C 程序}
	我们从一个 C 语言中“交换两个整数”的子程序开始:
	\lstcode{xchg1.c}
	\pause
	\begin{itemize}
		\item \lstinline{*} 和 \lstinline{&} 之类的东西很多……
	\end{itemize}
\end{frame}

\begin{frame}{引用类型}
	C++ 引入了引用类型,它基本就是指针套壳:
	\lstcode{xchg2.cc}
	\pause
	\begin{itemize}
		\item 只能交换 \lstinline{int}
	\end{itemize}
\end{frame}

\begin{frame}{函数重载}
	重载允许我们编写同名,但参数类型不同的子程序:
	\lstcode{xchg3.cc}
\end{frame}

\begin{frame}{例:绝对值函数}
	\lstcode{abs.cc}
	输出为:
	\lstterm{abs.out}
\end{frame}

\begin{frame}{函数模板}
	模板使得编译器为多种类型自动生成重载函数:
	\lstcode{xchg4.cc}
\end{frame}

\begin{frame}{命名空间}
	命名空间可以用于防止名称冲突:
	\lstcode{namespace.cc}
\end{frame}

\begin{frame}{std 命名空间}
	C++ 标准库中的所有名称都位于 \lstinline{std} 命名空间中,
	所以很多竞赛程序都会直接导入其中的所有名称以便使用:
	\lstcode{using-std.cc}
	但是这在写软件项目时通常是不允许的,也有一些竞赛选手不喜欢这样做。
\end{frame}

\begin{frame}{标准库的排序函数}
	\lstcode{sort.cc}
	输出为:
	\lstterm{sort.out}
	如无特殊需要,比赛时不要写任何排序算法!
\end{frame}

\begin{frame}{二分查找函数}
	\lstcode{bsrch.cc}
	输出为:
	\lstterm{bsrch.out}
	区间 $[l, r)$ 包含了所有的 $4$。
\end{frame}

\begin{frame}{重载运算符}
	\lstcode{sort-vec.cc}
\end{frame}

\begin{frame}{重载运算符}
	输出为:
	\lstterm{sort-vec.out}
\end{frame}

\begin{frame}{set}
	\lstinline{std::set} 是一种基于有序数据结构的 (具体来说是二叉搜索树)
	集合类型。
	\lstcode{set.cc}
	\lstterm{set.out}
\end{frame}

\begin{frame}{动态插入和删除}
	\lstcode{set2.cc}
	\lstterm{set2.out}
\end{frame}

\begin{frame}{二分查找}
	\lstcode{set3.cc}
	\lstterm{set3.out}
\end{frame}

\begin{frame}{multiset}
	和 \lstinline{set} 类似,但是允许重复元素:
	\lstcode{multiset.cc}
	\lstterm{multiset.out}
\end{frame}

\begin{frame}{map}
	除了集合以外,还维护该集合上的一个映射:
	\lstcode{map.cc}
	\lstterm{map.out}
\end{frame}

{\xdbg%末页致谢
\begin{frame}[plain,noframenumbering]
 \finalpage{{\huge GL and HF!}}
\end{frame}}

\end{document}
